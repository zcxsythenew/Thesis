%---------------------------------------------------------------------------%
%->> Backmatter
%---------------------------------------------------------------------------%
\chapter[致谢]{致\quad 谢}\chaptermark{致\quad 谢}% syntax: \chapter[目录]{标题}\chaptermark{页眉}
%\thispagestyle{noheaderstyle}% 如果需要移除当前页的页眉
%\pagestyle{noheaderstyle}% 如果需要移除整章的页眉

在国科大和计算所的三年硕士研究生的生活已经到达了尾声。在这三年之间,我不仅接触、学习了专业知识,更重要的是锻炼了发现、解决科学问题的能力和表达能力。

首先,我要感谢我的导师姜海洋老师,姜老师在百忙之中抽出时间对我的工作和论文进行悉心指导,从培养、选题到写作,都给予了我很大的帮助。此外,张广兴老师和刁祖龙老师对我在论文工作中存在的疑问、困难都给予了解答,为我指明了改进的方向。

同时,我要感谢我的同学和师兄弟姐妹,我从你们这里接触到许多不同领域方向的科研资讯。感谢我的舍友们,你们让我的在校生活充满了色彩。

最后,我要感谢我的家人,他们一直以来对我的理解、支持和鼓励是我前进的动力。


\rightline{2024年6月}
\chapter{作者简历及攻读学位期间发表的学术论文与其他相关学术成果}

\section*{作者简历:}
2017年08月——2021年06月,在中山大学计算机学院获得工学学士学位。

2021年09月——2024年07月,在中国科学院计算技术研究所攻读工学硕士学位。

% 工作经历:

% \section*{已发表(或正式接受)的学术论文:}

% {
% \setlist[enumerate]{}% restore default behavior
% \begin{enumerate}[nosep]
%     \item 已发表的工作1
%     \item 已发表的工作2
% \end{enumerate}
% }

% \section*{申请或已获得的专利:}

% (无专利时此项不必列出)

% \section*{参加的研究项目及获奖情况:}


\cleardoublepage[plain]% 让文档总是结束于偶数页,可根据需要设定页眉页脚样式,如 [noheaderstyle]
%---------------------------------------------------------------------------%
