\chapter{总结与展望}{
{
\let\cleardoublepage\relax
}

\section{本文工作总结}

随着容器化技术的发展,容器化集群越来越广泛地得到应用。然而,与此同时,容器化集群也得到了越来越多攻击者的青睐。因此,有必要采取措施,及时检测攻击行为。

容器化攻击分为侦察、立足、横向移动、攻击、清理五个阶段。在这些阶段中,横向移动是检测攻击行为的最重要的阶段,也最具有可行性。在容器化环境中,网络流量最容易获得。本文对容器化环境中基于网络流量的横向移动检测方法进行了研究。

本文回顾了现有的横向移动检测工作。这些工作大多数在企业网络环境下进行。一些工作强依赖于用户行为信息、系统事件等,无法应用于容器化环境。其他大多数工作基于用户登录信息采用了动态图的链路预测方法,从静态图到动态图,方法的准确性逐步提高;然而这些基于图的方法也不适用于容器化环境,因为在容器化环境中,没有用户登录信息可用。随后,本文对容器化环境中的可行的横向移动检测技术路线进行了讨论,一些方法分别采用了编码器—解码器结构、对抗训练、引入 Transformer 模型等手段,为本文提出的方法提供了思路。

为了解决检测特征不明确的问题,在网络流量特征方面,本文进行了网络流量特征分析和特征重要度评估。通过特征分析,本文发现通过对前向 bulk 等相关特征的挖掘,可以直接检测到关键的横向移动流,这些流代表攻击者利用被攻击的负载向容器化集群的 API 服务器的恶意访问。通过特征重要度评估,本文得到了网络流量各特征的重要度,供后续实验进行特征筛选使用。

在时空特性方面,本文进行了网络流量拓扑分析和特征嵌入方法研究。通过拓扑结构分析,本文发现通过 API 服务器与负载之间的通信,可以直接发现关键的横向移动流,这些流代表负载在攻击者的控制下,对 API 服务器进行恶意操作。然而,这些流只占横向移动流的一小部分。为了检测到大多数的横向移动流,本文通过图神经网络方法和循环神经网络方法,将网络流量转换为节点的嵌入向量和时间特征向量,供后续模型使用。

为了提高横向移动流量的检测率、降低误报率,本文提出了基于 Transformer 的两阶段横向移动检测方法。在第一阶段,基于最值和拓扑的横向移动检测利用了特征分析和拓扑分析的结果,检测最关键的横向移动行为;在第二个阶段,基于 Transformer 的横向移动检测模型 LMDCE 包含了空间特征嵌入模块、时间特征嵌入模块和编码器—解码器模块,并将现有工作的对抗训练机制改进两步预测机制,提升了模型的检测率并降低了误报率,达到了 0.9640 的 AUC 分数,优于其他现有模型。本文还对模型的各模块的相对重要性进行了评估,并讨论了异常阈值选定的问题。最后,本文对方法的两个阶段进行了集成验证。因此,本文提出的容器化环境中的横向移动检测方法是有效的。

\section{后续展望}

本文未来的发展方向包括:

\begin{enumerate}
    \item 除了 Kubernetes-dataset 提供的一种横向移动场景以外,还需要在真实场景中收集其他横向移动场景的流量,以便对本文所提出的方法进行交叉验证。
    \item 本文研究的容器化集群规模还不够大,在更大规模的容器化环境中的横向移动检测有待研究。
    \item 由于引入了 Transformer 结构,并包含时间特征嵌入模块,本文提出的方法的空间开销较大,需研究减少开销的方法。
    \item 使用本文提出的检测方法,开发容器化集群中横向移动的在线检测系统。
\end{enumerate}
}