%---------------------------------------------------------------------------%
%->> Frontmatter
%---------------------------------------------------------------------------%
%-
%-> 生成封面
%-

\maketitle% 生成中文封面
\MAKETITLE% 生成英文封面
%-
%-> 作者声明
%-
\makedeclaration% 生成声明页
%-
%-> 中文摘要
%-
\intobmk\chapter*{摘\quad 要}% 显示在书签但不显示在目录
\setcounter{page}{1}% 开始页码
\pagenumbering{Roman}% 页码符号

云计算技术的持续进步使得云原生容器平台成为企业数字化转型的核心。然而,随着容器化集群的广泛部署,许多集群受到各种攻击的困扰,安全性问题日益凸显。这些攻击可以分为若干个阶段,其中横向移动阶段的检测十分重要,在这一阶段,攻击者已进入集群中,开始收集集群内的信息,并寻找集群内的下一个攻击目标。尽管对于横向移动的检测在企业网络等环境中已有所研究,但针对容器化环境的检测方法仍然不足。

目前,容器化环境下的横向移动检测存在两方面的挑战:一是与已有研究的环境不同,容器化集群中的横向移动通常不包含用户身份验证信息,使得可用于检测的特征不明确;二是横向移动检测任务的横向移动检测任务的分类难度大,且样本不均衡,对高检测率和低误报率提出了高要求。

为了解决上述挑战,本文通过特征分析、特征重要度评估和特征嵌入方法对数据进行了增强,并提出了一种横向移动检测方法,提高了容器化环境中的检测性能。本文主要工作如下:

\begin{itemize}
    \item 针对容器化集群可用于检测的特征不明确的问题,本文从两个方面着手分析。在网络流量特征方面,本文进行了特征分析和特征重要度评估。通过数据集验证表明,通过特征分析,可发现横向移动流量中的``关键少数'',它们代表了攻击者在集群中部署恶意脚本并窃取机密信息的行为;通过特征重要度评估,可对重要性最低的特征进行筛除,节省 27\% 的模型训练时间。
    \item 在网络流量的时空特性方面,本文进行了拓扑分析和特征嵌入方法研究。通过拓扑分析,可以发现横向移动使网络拓扑发生改变,受感染的负载向 API 服务器非法通信的同时与外部网络保持联系;通过基于图神经网络的空间特征嵌入方法和基于循环神经网络的时间特征嵌入方法,可分别使模型的 AUC 分数提升 4 个和 3 个百分点。
    \item 针对横向移动检测任务的检测性能问题,本文提出了一种横向移动检测方法。该方法分两个阶段进行检测:首先,通过基于最值和拓扑的检测方法,本文从数据集中检出了关键的横向移动流量。在第二个阶段,提出了 LMDCE 模型,通过基于 Transformer 模型的编码器—解码器模块和两步预测机制,提升了检测的准确率,达到了 96.401\% 的 AUC 分数,超过了现有其他横向移动检测模型在容器化环境下的表现。
\end{itemize}

\keywords{容器化,横向移动,Transformer,自动编码器,入侵检测系统}% 中文关键词
%-
%-> 英文摘要
%-
\intobmk\chapter*{Abstract}% 显示在书签但不显示在目录

Continuous advances in cloud computing technology have made cloud-native containerized platforms the core of enterprise digital transformation. However, with the widespread deployment of containerized clusters, many clusters are plagued by various attacks, and security issues are becoming increasingly prominent. These attacks can be categorized into several phases, among which the detection of the lateral movement phase is very important. In this phase, the attacker has entered the cluster, starts to collect information within the cluster and looks for the next attack target within the cluster. Although lateral movement detection has been studied in environments such as enterprise networks, detection methods for containerized environments are still insufficient.

Currently, there are two challenges for lateral movement detection in containerized environments. First, unlike the environments that have been researched, lateral movements in containerized clusters usually do not contain user authentication information, making the features that can be used for detection unclear. Second, the difficulty of classifying lateral movement detection tasks with uneven samples puts a high demand on high detection rate and low false alarm rate.

To address the above challenges, this thesis enhances the data through feature analysis, feature importance evaluation and feature embedding methods, and proposes a lateral movement detection model to improve the detection performance in containerized environments. The main work of this thesis is as follows:

\begin{enumerate}
    \item Aiming at the problem that the features available for detection in containerized clusters are not clear, this thesis starts to analyze from two aspects. In terms of network flow features, this thesis performs feature analysis and feature importance assessment. Validation of the dataset shows that the feature analysis can identify the ``critical few'' in the lateral movement flows, which represent the attacker's behavior of deploying malicious scripts and stealing confidential information in the cluster; and the feature importance assessment can filter out the features with the lowest importance, which saves 27\% of the model training time. 

    \item In terms of the spatial and temporal characteristics of network flows, this thesis carries out research on topology analysis and feature embedding methods. Through topology analysis, it can be found that lateral movement changes the network topology, and the infected pod communicates illegally to the API server while keeping in touch with the external network; through the spatial feature embedding method based on graph neural network and the temporal feature embedding method based on recurrent neural network, the AUC scores of the model can be improved by 4 and 3 percentage points, respectively.

    \item Aiming at the detection performance problem of lateral movement detection task, this thesis proposes a lateral movement detection method. The method performs detection in two stages. First, through the detection method based on the most value and topology, this thesis detects critical lateral movement flows from the dataset. Second, this thesis proposes a model called LMDCE; the accuracy of detection is improved by a Transformer model-based encoder-decoder module and a two-phase prediction mechanism, achieving an AUC score of 96.401\%, which outperforms the performance of other existing lateral-movement detection models in a containerized environment.
\end{enumerate}
    %- the current style, comment all the lines in plain style definition.

\KEYWORDS{Containerization, Lateral Movement, Transformer, Autoencoder, Intrusion Detection System}% 英文关键词

\pagestyle{enfrontmatterstyle}%
\cleardoublepage\pagestyle{frontmatterstyle}%

%---------------------------------------------------------------------------%
