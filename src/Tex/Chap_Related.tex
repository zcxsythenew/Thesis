\chapter{相关研究成果与进展}{
{
\let\cleardoublepage\relax
}

本章首先介绍现有的横向移动检测工作,并进一步探讨该类工作所主要采用的动态图链路预测方法。然后,最后介绍本文所采用的时间序列异常检测方法的相关工作。



\section{数据集描述}

为了对容器化环境中的网络流量特征和横向移动检测方法进行研究,本文采用了 Kubernetes-dataset 数据集\citep{sever2023kubernetes}作为研究对象。

\section{本章小结}

本章首先分析并归纳了现有的横向移动检测模型,大多数模型基于图的链路预测实现。然而这些模型不适用于容器化的场景:容器化的环境中更需要考虑全局的流量信息,而不是图上每两个节点之间的局部信息。因此,本章的第二节总结了时间序列异常检测方法,这些方法有很多可采取之处,包括基于编码器—解码器的重建、对抗训练、Transformer 结构的引入,为本研究提供了思路。
}